\documentclass{article}
\usepackage{colortbl}
\usepackage{multirow}
\usepackage{graphicx} % For rotating text
\usepackage{makecell} % For vertical text
\usepackage{array} % For adjusting column width
\usepackage{xcolor} % For defining custom colors
\usepackage{tikz} % For heat map coloring
\usepackage{textgreek} % For Greek letters
\usepackage{afterpage} % For separate page
\usepackage{hyperref} % For hyperlinks

% Define custom colors for circles and families
\definecolor{circle-green}{RGB}{102, 204, 102}
\definecolor{circle-red}{RGB}{204, 102, 102}
\definecolor{circle-yellow}{RGB}{255, 204, 0}
\definecolor{family-vaes}{RGB}{231, 180, 229}
\definecolor{family-gans}{RGB}{166, 208, 255}
\definecolor{family-flows}{RGB}{199, 230, 204}
\definecolor{family-autoregressive}{RGB}{255, 213, 153}
\definecolor{family-hybrid}{RGB}{191, 191, 191}
\definecolor{family-diffusion}{RGB}{255, 191, 128}
\definecolor{family-other}{RGB}{229, 204, 255}

% Define circular cell command with colored circles
\newcommand{\heatmapcell}[1]{\begin{tikzpicture}\node[draw,circle,minimum size=0.8cm,fill=#1,text=black] {};\end{tikzpicture}}

% Define family command with transparent background
\newcommand{\family}[2]{\begin{tikzpicture}[baseline={(N.base)}]\node[rectangle,rounded corners=3pt,inner sep=1pt,fill=#1,text=white,text width=5cm]{\Large \textbf{#2}};\end{tikzpicture}}

% Define hyperlink highlighting without boxes
\hypersetup{
  colorlinks=true,
  linkcolor=blue,
  urlcolor=cyan
}

\title{\textbf{Application Matrix of Families of Generative Models for Computer Vision}}
\author{Ed Mwanza, PhD candidate Comp. Sci.}

\begin{document}

\maketitle

\afterpage{%
\clearpage
\thispagestyle{empty}

\textcolor{family-vaes}{\family{family-vaes}{Variational Autoencoders (VAEs)}:}
\begin{itemize}
  \item Vanilla VAE
    \begin{itemize}
      \item \textbf{Repository Link}: \href{https://github.com/username/vanilla-vae}{GitHub Repository}
      \item \textbf{Paper Link}: \href{https://arxiv.org/abs/1234.5678}{Example Paper}
      \item \textbf{Owner}: Company/Group Name
      \item \textbf{Explanation}: Vanilla VAE is a basic VAE model that learns a latent representation of data.
    \end{itemize}
  \item $\beta$-VAE
    \begin{itemize}
      \item \textbf{Repository Link}: \href{https://github.com/username/beta-vae}{GitHub Repository}
      \item \textbf{Paper Link}: \href{https://arxiv.org/abs/2345.6789}{Example Paper}
      \item \textbf{Owner}: Company/Group Name
      \item \textbf{Explanation}: $\beta$-VAE introduces a regularization term to the VAE's objective function to control disentanglement of latent factors.
    \end{itemize}
  \item VQ-VAE
    \begin{itemize}
      \item \textbf{Repository Link}: \href{https://github.com/username/vq-vae}{GitHub Repository}
      \item \textbf{Paper Link}: \href{https://arxiv.org/abs/3456.7890}{Example Paper}
      \item \textbf{Owner}: Company/Group Name
      \item \textbf{Explanation}: VQ-VAE uses vector quantization to learn a discrete latent space and enables high-quality image generation.
    \end{itemize}
  % Add other VAE models with their respective sub-bulletins
  \item VAE-GAN
    \begin{itemize}
      \item \textbf{Repository Link}: \href{https://github.com/username/vae-gan}{GitHub Repository}
      \item \textbf{Paper Link}: \href{https://arxiv.org/abs/7890.1234}{Example Paper}
      \item \textbf{Owner}: Company/Group Name
      \item \textbf{Explanation}: VAE-GAN combines the variational autoencoder (VAE) and generative adversarial network (GAN) frameworks for improved generative modeling.
    \end{itemize}
  \item CVAE (Conditional VAE)
    \begin{itemize}
      \item \textbf{Repository Link}: \href{https://github.com/username/cvae}{GitHub Repository}
      \item \textbf{Paper Link}: \href{https://arxiv.org/abs/9012.3456}{Example Paper}
      \item \textbf{Owner}: Company/Group Name
      \item \textbf{Explanation}: CVAE extends the VAE framework to incorporate conditional information for controlled generation.
    \end{itemize}
  \item DFC-VAE (Disentangled Feature Control VAE)
    \begin{itemize}
      \item \textbf{Repository Link}: \href{https://github.com/username/dfc-vae}{GitHub Repository}
      \item \textbf{Paper Link}: \href{https://arxiv.org/abs/6789.0123}{Example Paper}
      \item \textbf{Owner}: Company/Group Name
      \item \textbf{Explanation}: DFC-VAE introduces mechanisms to disentangle specific features in the latent space for better control over generated outputs.
    \end{itemize}
  \item HiVAE (Hierarchical VAE)
    \begin{itemize}
      \item \textbf{Repository Link}: \href{https://github.com/username/hivae}{GitHub Repository}
      \item \textbf{Paper Link}: \href{https://arxiv.org/abs/3456.7890}{Example Paper}
      \item \textbf{Owner}: Company/Group Name
      \item \textbf{Explanation}: HiVAE incorporates hierarchical structure into the VAE framework to capture hierarchical relationships in data.
    \end{itemize}
  \item VLAE (Variational Lossy Autoencoder)
    \begin{itemize}
      \item \textbf{Repository Link}: \href{https://github.com/username/vlae}{GitHub Repository}
      \item \textbf{Paper Link}: \href{https://arxiv.org/abs/9012.3456}{Example Paper}
      \item \textbf{Owner}: Company/Group Name
      \item \textbf{Explanation}: VLAE is an extension of the VAE model that allows lossy compression of data by learning multiple latent representations.
    \end{itemize}
  \item AdaVAE (Adaptive VAE)
    \begin{itemize}
      \item \textbf{Repository Link}: \href{https://github.com/username/adavae}{GitHub Repository}
      \item \textbf{Paper Link}: \href{https://arxiv.org/abs/6789.0123}{Example Paper}
      \item \textbf{Owner}: Company/Group Name
      \item \textbf{Explanation}: AdaVAE incorporates an adaptive mechanism to dynamically adjust the capacity of the VAE during training.
    \end{itemize}
  \item SCVAE (Semi-Supervised Conditional VAE)
    \begin{itemize}
      \item \textbf{Repository Link}: \href{https://github.com/username/scvae}{GitHub Repository}
      \item \textbf{Paper Link}: \href{https://arxiv.org/abs/3456.7890}{Example Paper}
      \item \textbf{Owner}: Company/Group Name
      \item \textbf{Explanation}: SCVAE combines the conditional VAE framework with semi-supervised learning for enhanced modeling of labeled and unlabeled data.
    \end{itemize}
  \item AAE (Adversarial Autoencoder)
    \begin{itemize}
      \item \textbf{Repository Link}: \href{https://github.com/username/aae}{GitHub Repository}
      \item \textbf{Paper Link}: \href{https://arxiv.org/abs/9012.3456}{Example Paper}
      \item \textbf{Owner}: Company/Group Name
      \item \textbf{Explanation}: AAE combines the autoencoder and GAN frameworks to achieve unsupervised learning and adversarial training.
    \end{itemize}
\end{itemize}

\textcolor{family-gans}{\family{family-gans}{Generative Adversarial Networks (GANs)}:}
\begin{itemize}
  \item Vanilla GAN
    \begin{itemize}
      \item \textbf{Repository Link}: \href{https://github.com/username/vanilla-gan}{GitHub Repository}
      \item \textbf{Paper Link}: \href{https://arxiv.org/abs/2345.6789}{Example Paper}
      \item \textbf{Owner}: Company/Group Name
      \item \textbf{Explanation}: Vanilla GAN is the original GAN model that consists of a generator and discriminator network.
    \end{itemize}
  \item DCGAN (Deep Convolutional GAN)
    \begin{itemize}
      \item \textbf{Repository Link}: \href{https://github.com/username/dcgan}{GitHub Repository}
      \item \textbf{Paper Link}: \href{https://arxiv.org/abs/3456.7890}{Example Paper}
      \item \textbf{Owner}: Company/Group Name
      \item \textbf{Explanation}: DCGAN extends the GAN model with deep convolutional networks for more stable training and better image generation.
    \end{itemize}
  \item CGAN (Conditional GAN)
    \begin{itemize}
      \item \textbf{Repository Link}: \href{https://github.com/username/cgan}{GitHub Repository}
      \item \textbf{Paper Link}: \href{https://arxiv.org/abs/9012.3456}{Example Paper}
      \item \textbf{Owner}: Company/Group Name
      \item \textbf{Explanation}: CGAN incorporates conditional information into the GAN framework for controlled generation.
    \end{itemize}
  \item WGAN (Wasserstein GAN)
    \begin{itemize}
      \item \textbf{Repository Link}: \href{https://github.com/username/wgan}{GitHub Repository}
      \item \textbf{Paper Link}: \href{https://arxiv.org/abs/6789.0123}{Example Paper}
      \item \textbf{Owner}: Company/Group Name
      \item \textbf{Explanation}: WGAN uses Wasserstein distance as the training objective for improved stability and meaningful loss metrics.
    \end{itemize}
  \item LSGAN (Least Squares GAN)
    \begin{itemize}
      \item \textbf{Repository Link}: \href{https://github.com/username/lsgan}{GitHub Repository}
      \item \textbf{Paper Link}: \href{https://arxiv.org/abs/3456.7890}{Example Paper}
      \item \textbf{Owner}: Company/Group Name
      \item \textbf{Explanation}: LSGAN replaces the binary GAN loss with a least squares loss for better training dynamics and improved image quality.
    \end{itemize}
  \item CycleGAN
    \begin{itemize}
      \item \textbf{Repository Link}: \href{https://github.com/username/cyclegan}{GitHub Repository}
      \item \textbf{Paper Link}: \href{https://arxiv.org/abs/9012.3456}{Example Paper}
      \item \textbf{Owner}: Company/Group Name
      \item \textbf{Explanation}: CycleGAN performs image-to-image translation using cycle consistency loss to learn mappings between domains without paired data.
    \end{itemize}
  \item ProGAN (Progressive GAN)
    \begin{itemize}
      \item \textbf{Repository Link}: \href{https://github.com/username/progan}{GitHub Repository}
      \item \textbf{Paper Link}: \href{https://arxiv.org/abs/6789.0123}{Example Paper}
      \item \textbf{Owner}: Company/Group Name
      \item \textbf{Explanation}: ProGAN progressively grows both the generator and discriminator during training to generate high-resolution images.
    \end{itemize}
\end{itemize}

\textcolor{family-flows}{\family{family-flows}{Flow-based Models}:}
\begin{itemize}
  \item RealNVP (Real-valued Non-Volume Preserving)
    \begin{itemize}
      \item \textbf{Repository Link}: \href{https://github.com/username/realnvp}{GitHub Repository}
      \item \textbf{Paper Link}: \href{https://arxiv.org/abs/3456.7890}{Example Paper}
      \item \textbf{Owner}: Company/Group Name
      \item \textbf{Explanation}: RealNVP is a flow-based model that models the data distribution with a sequence of invertible transformations.
    \end{itemize}
  \item Glow
    \begin{itemize}
      \item \textbf{Repository Link}: \href{https://github.com/username/glow}{GitHub Repository}
      \item \textbf{Paper Link}: \href{https://arxiv.org/abs/9012.3456}{Example Paper}
      \item \textbf{Owner}: Company/Group Name
      \item \textbf{Explanation}: Glow is a flow-based model that utilizes invertible $1\times1$ convolutions to learn a tractable and flexible data distribution.
    \end{itemize}
  \item FFJORD (Continuous-Time Flows)
    \begin{itemize}
      \item \textbf{Repository Link}: \href{https://github.com/username/ffjord}{GitHub Repository}
      \item \textbf{Paper Link}: \href{https://arxiv.org/abs/6789.0123}{Example Paper}
      \item \textbf{Owner}: Company/Group Name
      \item \textbf{Explanation}: FFJORD is a flow-based model that leverages continuous-time normalizing flows for efficient and expressive generative modeling.
    \end{itemize}
\end{itemize}

\textcolor{family-autoregressive}{\family{family-autoregressive}{Auto-regressive Models}:}
\begin{itemize}
  \item PixelRNN
    \begin{itemize}
      \item \textbf{Repository Link}: \href{https://github.com/username/pixelrnn}{GitHub Repository}
      \item \textbf{Paper Link}: \href{https://arxiv.org/abs/4567.8901}{Example Paper}
      \item \textbf{Owner}: Company/Group Name
      \item \textbf{Explanation}: PixelRNN is an auto-regressive model that generates images by modeling the conditional distribution of each pixel given previous pixels.
    \end{itemize}
  \item PixelCNN
    \begin{itemize}
      \item \textbf{Repository Link}: \href{https://github.com/username/pixelcnn}{GitHub Repository}
      \item \textbf{Paper Link}: \href{https://arxiv.org/abs/7890.1234}{Example Paper}
      \item \textbf{Owner}: Company/Group Name
      \item \textbf{Explanation}: PixelCNN is an auto-regressive model that generates images by modeling the joint distribution of all pixels in parallel.
    \end{itemize}
  \item WaveNet
    \begin{itemize}
      \item \textbf{Repository Link}: \href{https://github.com/username/wavenet}{GitHub Repository}
      \item \textbf{Paper Link}: \href{https://arxiv.org/abs/9012.3456}{Example Paper}
      \item \textbf{Owner}: Company/Group Name
      \item \textbf{Explanation}: WaveNet is an auto-regressive model primarily used for speech and audio generation, but can also be applied to image generation.
    \end{itemize}
\end{itemize}

\textcolor{family-hybrid}{\family{family-hybrid}{Hybrid Models}:}
\begin{itemize}
  \item VQ-VAE-2 (Vector Quantized VAE)
    \begin{itemize}
      \item \textbf{Repository Link}: \href{https://github.com/username/vq-vae-2}{GitHub Repository}
      \item \textbf{Paper Link}: \href{https://arxiv.org/abs/5678.9012}{Example Paper}
      \item \textbf{Owner}: Company/Group Name
      \item \textbf{Explanation}: VQ-VAE-2 combines a VAE with vector quantization to learn a discrete latent space representation.
    \end{itemize}
  \item CVAE-GAN (Conditional VAE-GAN)
    \begin{itemize}
      \item \textbf{Repository Link}: \href{https://github.com/username/cvae-gan}{GitHub Repository}
      \item \textbf{Paper Link}: \href{https://arxiv.org/abs/6789.0123}{Example Paper}
      \item \textbf{Owner}: Company/Group Name
      \item \textbf{Explanation}: CVAE-GAN combines the conditional VAE and GAN frameworks to achieve controlled generation with disentangled latent variables.
    \end{itemize}
  \item VAE-Glow
    \begin{itemize}
      \item \textbf{Repository Link}: \href{https://github.com/username/vae-glow}{GitHub Repository}
      \item \textbf{Paper Link}: \href{https://arxiv.org/abs/3456.7890}{Example Paper}
      \item \textbf{Owner}: Company/Group Name
      \item \textbf{Explanation}: VAE-Glow combines a VAE with the Glow flow-based model for improved generative modeling.
    \end{itemize}
\end{itemize}

\textcolor{family-diffusion}{\family{family-diffusion}{Diffusion Models}:}
\begin{itemize}
  \item Noise-Contrastive Estimation (NCE)
    \begin{itemize}
      \item \textbf{Repository Link}: \href{https://github.com/username/noise-contrastive-estimation}{GitHub Repository}
      \item \textbf{Paper Link}: \href{https://arxiv.org/abs/7890.1234}{Example Paper}
      \item \textbf{Owner}: Company/Group Name
      \item \textbf{Explanation}: NCE is a diffusion-based model that estimates the data distribution by contrasting it with a noise distribution.
    \end{itemize}
  \item Diffusion Probabilistic Models
    \begin{itemize}
      \item \textbf{Repository Link}: \href{https://github.com/username/diffusion-probabilistic-models}{GitHub Repository}
      \item \textbf{Paper Link}: \href{https://arxiv.org/abs/9012.3456}{Example Paper}
      \item \textbf{Owner}: Company/Group Name
      \item \textbf{Explanation}: Diffusion probabilistic models use a series of diffusion steps to model the data distribution effectively.
    \end{itemize}
\end{itemize}

\textcolor{family-other}{\family{family-other}{Other notable models}:}
\begin{itemize}
  \item Adversarial Autoencoders
    \begin{itemize}
      \item \textbf{Repository Link}: \href{https://github.com/username/adversarial-autoencoders}{GitHub Repository}
      \item \textbf{Paper Link}: \href{https://arxiv.org/abs/8901.2345}{Example Paper}
      \item \textbf{Owner}: Company/Group Name
      \item \textbf{Explanation}: Adversarial Autoencoders combine the adversarial training of GANs with the autoencoder framework for unsupervised learning and generation.
    \end{itemize}
  \item StyleGAN (Style-Generative Adversarial Network)
    \begin{itemize}
      \item \textbf{Repository Link}: \href{https://github.com/username/stylegan}{GitHub Repository}
      \item \textbf{Paper Link}: \href{https://arxiv.org/abs/3456.7890}{Example Paper}
      \item \textbf{Owner}: Company/Group Name
      \item \textbf{Explanation}: StyleGAN generates high-quality images with fine-grained control over the style and attributes of the generated content.
    \end{itemize}
  \item BigGAN
    \begin{itemize}
      \item \textbf{Repository Link}: \href{https://github.com/username/biggan}{GitHub Repository}
      \item \textbf{Paper Link}: \href{https://arxiv.org/abs/9012.3456}{Example Paper}
      \item \textbf{Owner}: Company/Group Name
      \item \textbf{Explanation}: BigGAN is a large-scale GAN model capable of generating high-resolution images with improved diversity and quality.
    \end{itemize}
\end{itemize}

\clearpage
}

\begin{table}[!htbp]
  \centering
  \caption{Application Matrix of Families of Generative Models for Computer Vision}
  
  \vspace{1em}
  
  \resizebox{\textwidth}{!}{
    \begin{tabular}{|l|*{7}{>{\centering\arraybackslash}m{1.6cm}|}} % Adjust column width as needed
      \hline
      \multicolumn{1}{|c|}{\multirow{2}{*}{\textbf{}}} & \multicolumn{7}{c|}{\textbf{Applications}} \\
      \cline{2-8}
      & \textbf{\textcolor{black}{Data Augmentation}} & \textbf{\textcolor{black}{Super Resolution}} & \textbf{\textcolor{black}{Inpainting}} & \textbf{\textcolor{black}{Denoising}} & \textbf{\textcolor{black}{Style Transfer}} & \textbf{\textcolor{black}{Object Transfiguration}} & \textbf{\textcolor{black}{Image Colorization}} \\
      \hline
      \textcolor{family-vaes}{\family{family-vaes}{\textcolor{black}{VAEs}}} & \heatmapcell{circle-green} & \heatmapcell{circle-red} & \heatmapcell{circle-red} & \heatmapcell{circle-red} & \heatmapcell{circle-red} & \heatmapcell{circle-red} & \heatmapcell{circle-red} \\
      \hline
      \textcolor{family-gans}{\family{family-gans}{\textcolor{black}{GANs}}} & \heatmapcell{circle-green} & \heatmapcell{circle-red} & \heatmapcell{circle-red} & \heatmapcell{circle-red} & \heatmapcell{circle-green} & \heatmapcell{circle-green} & \heatmapcell{circle-red} \\
      \hline
      \textcolor{family-flows}{\family{family-flows}{\textcolor{black}{Flow-based Models}}} & \heatmapcell{circle-green} & \heatmapcell{circle-red} & \heatmapcell{circle-red} & \heatmapcell{circle-red} & \heatmapcell{circle-yellow} & \heatmapcell{circle-yellow} & \heatmapcell{circle-red} \\
      \hline
      \textcolor{family-autoregressive}{\family{family-autoregressive}{\textcolor{black}{Auto-regressive Models}}} & \heatmapcell{circle-green} & \heatmapcell{circle-red} & \heatmapcell{circle-red} & \heatmapcell{circle-red} & \heatmapcell{circle-yellow} & \heatmapcell{circle-yellow} & \heatmapcell{circle-red} \\
      \hline
      \textcolor{family-hybrid}{\family{family-hybrid}{\textcolor{black}{Hybrid Models}}} & \heatmapcell{circle-green} & \heatmapcell{circle-red} & \heatmapcell{circle-red} & \heatmapcell{circle-red} & \heatmapcell{circle-green} & \heatmapcell{circle-green} & \heatmapcell{circle-red} \\
      \hline
      \textcolor{family-diffusion}{\family{family-diffusion}{\textcolor{black}{Diffusion Models}}} & \heatmapcell{circle-green} & \heatmapcell{circle-red} & \heatmapcell{circle-red} & \heatmapcell{circle-red} & \heatmapcell{circle-red} & \heatmapcell{circle-red} & \heatmapcell{circle-red} \\
      \hline
      \textcolor{family-other}{\family{family-other}{\textcolor{black}{Other notable models}}} & \heatmapcell{circle-green} & \heatmapcell{circle-green} & \heatmapcell{circle-green} & \heatmapcell{circle-green} & \heatmapcell{circle-green} & \heatmapcell{circle-green} & \heatmapcell{circle-green} \\
      \hline
    \end{tabular}
  }
\end{table}

\end{document}
